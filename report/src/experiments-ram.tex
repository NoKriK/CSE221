\newpage
\subsection{Random Access Memory}

\subsubsection{RAM access time}
\paragraph{Methodology}
We measured the back-to-back-load RAM access latency, because it is well accepted by most software developers and system researchers.
We used the method described in the paper \emph{lmbench: Portable Tools for Performance Analysis} to measure the memory and cache latency.
We created arrays of different sizes, then we created a list of pointers to walk
the list and then we walked the list like this :
\begin{verbatim}
p = *p;
\end{verbatim}

We created the list with different strides and did the same measurement.
For each couple of stride/array sizes we did 1,000,000 iteration through the list.
The value of the pointer p is then printed to avoid that the compiler optimize the loop and remove the instruction.

We measured the value of different sized arrays between 512KB and 128MB.
The stride varied between 1 and 15 integer.

\paragraph{Predictions}
According to the \emph{Intel® 64 and IA-32 Architectures Optimization Reference Manual}
\cite{intel-archi-opti-intel64} base case latencies are :
\begin{description}
\item[L1 cache] 4 cycles
\item[L2 cache] 12 cycles
\item[L3 cache] 28 cycles
\item[Main memory] 45 cycles
\end{description}
According to the RAM specification the latency is about 15ns.
It's about 45 cpu cycles as the CPU is running at 3.3Ghz.
Because of the way our measurement is setup, there is no
software cost involved in this measurement.


\paragraph{Results}
We are going to present the result table for stride 11
and the plotted graph for different strides from 1 to 15
and different array sizes.

\begin{table}[t!]
\begin{center}
\begin{tabular}{| r | l | l | l | r |}
\hline
Size of array 	& Hardware cost 	& Software cost 	& Prediction 	& Measured \\ \hline
512B 		&	4 cycles	&	0 cycle		&	4 cycles	&3.895 cycles	\\ \hline
1KB 		&	4 cycles	&	0 cycle		&	4 cycles	&3.886 cycles		\\ \hline
2KB 		&	4 cycles	&	0 cycle		&	4 cycles	&3.885 cycles		\\ \hline
4KB 		&	4 cycles	&	0 cycle		&	4 cycles	&3.886 cycles		\\ \hline
8KB 		&	4 cycles	&	0 cycle		&	4 cycles	&3.886 cycles		\\ \hline
16KB 		&	4 cycles	&	0 cycle		&	4 cycles	&3.888 cycles		\\ \hline
32KB 		&	4 cycles	&	0 cycle		&	4 cycles	&3.891 cycles		\\ \hline
64KB 		&	12 cycles	&	0 cycle		&	12 cycles	&5.949 cycles		\\ \hline
128KB 		&	12 cycles	&	0 cycle		&	12 cycles	&5.945 cycles		\\ \hline
256KB 		&	12 cycles	&	0 cycle		&	12 cycles	&6.209 cycles		\\ \hline
512KB 		&	28 cycles	&	0 cycle		&	28 cycles	&10.314 cycles		\\ \hline
1MB 		&	28 cycles	&	0 cycle		&	28 cycles	&10.399 cycles		\\ \hline
2MB 		&	28 cycles	&	0 cycle		&	28 cycles	&10.757 cycles		\\ \hline
4MB 		&	28 cycles	&	0 cycle		&	28 cycles	&15.585 cycles		\\ \hline
8MB 		&	45 cycles	&	0 cycle		&	45 cycles	&30.265 cycles		\\ \hline
16MB 		&	45 cycles	&	0 cycle		&	45 cycles	&37.458 cycles		\\ \hline
32MB 		&	45 cycles	&	0 cycle		&	45 cycles	&37.686 cycles		\\ \hline
64MB 		&	45 cycles	&	0 cycle		&	45 cycles	&37.092 cycles		\\ \hline
128MB 		&	45 cycles	&	0 cycle		&	45 cycles	&37.619 cycles		\\ \hline

\hline
\end{tabular}
\end{center}
\caption{RAM latency\label{tab:access-time}}
\end{table}

\begin{figure}[h!]
\begin{center}
\includegraphics[scale=0.8]{memoryLatencyImage}
\end{center}
\caption {RAM latency\label{fig:access-time}}

\end{figure}


The figure \ref{fig:access-time} and table \ref{tab:access-time} clearly
indicate the average latency for L1 cache, L2 cache, L3 cache and main memory.
When the stride is small, the reading time for arrays bigger than the L3 cache is still
about the same as L1 because those data are prefetched into the cache hiding the
latency.
Our prediction time is actually larger than the measured time.
It's probably because some pre-fetching is going on when we set up the reading
operation.
Also, the loop are partly unrolled by the compiler which improves performance.

\paragraph{Success of Methodology}
We successfully replicated the methodology of the lmbench paper.
We were able to distinguish the different cache levels.
Overall, this measurement is successful.



\subsubsection{RAM bandwidth}
\paragraph{Methodology}
We allocated an array of 128MB so that it doesn't fit into any cache (the L3 cache is 6MB).
To make the measurement, we walked through the array by incrementing a pointer to avoid summing an index and a base pointer on each iteration.
The array is filled once before the tests to ensure that the underlying pages are really allocated and to avoid TLB misses.

For the read bandwidth an integer value was read and added to an integer stored in a register.
The resulting value is printed after the measurement to avoid a compiler optimization.
For the write bandwidth, we filled the array of an arbitrary integer value.

The optimizations are turned on to reduce the number of operations not related to the memory read.
The only other operation are an incrementation of the pointer, a comparaison and a conditional jump.
All these operations are made on values stored in registers.
The compiler options -funroll-loops also helps avoid the overhead of the loop.

The measurement unit is in cycle for 128MB and is then translated to MB/s by calculating it with the CPU clock rate.
The result is averaged on 10000 iterations.

\paragraph{Predictions}
According to the specification of the RAM, the peak transfer rate is 10666 MB/s.
We are awaiting result wich should be near this value but doesn't reach this value as this is a theoritical value.
We have a negligible overhead in our measurement due to the loop operations.
We except the write bandwidth to be between 15\% and 30\% lower than read.
(It takes time  to charge or discharge the capacitor that is in each DRAM memory cell whereas
reading it is much faster.)

\paragraph{Results}
\begin{table}[h]
\begin{center}
\begin{tabular}{| l | l | l | l | l |}
\hline
Operation 	& Hardware cost & Software cost & Prediction 	& Measured \\
\hline
Read 128MB 	& 10000 MB/s	& 0 MB/s	& 10000 MB/s	& 9711.66 MB/s \\
\hline
Write 128MB 	& 8000 MB/s	& 0 MB/s	& 8000 MB/s	& 7575.58 MB/s \\
\hline
\end{tabular}
\end{center}
\caption{RAM bandwith\label{tab:ram-bandwidth}}
\end{table}

According to the table \ref{tab:ram-bandwidth} above,
we calculated the read tranfer rate is 9711 MB/s and the write transfer rate is 7575 MB/s.
As excepted, the results are near 10000 MB/s and the writing time is about 20\%
slower.

\paragraph{Success of Methodology}
Comparing to the specification of the RAM, our result is really close to the
ideal transfer rate.


\subsubsection{Page fault service time}
\paragraph{Methodology}
In order to manually create a page fault situation, we allocated two arrays of
the same size as the main memory.
Then we filled the first array to put it into main memory first and to be sure that the pages are allocated.
Afterwards, we filled the second array so that the first array will be totally paged out to the swap space.
The page fault service time is going to be the time it takes to get the value of an element from the first array.

We were only able to run the test three times.
Our test was taking too much time to run and our connection was timing out.
It was then impossible to reconnect to the virtual machine due to the whole
system swapping.

\paragraph{Predictions}
The latency to page in a page from the disks will be really high.

The request will travers many layers :
\begin{enumerate}
\item Page fault into the solaris operating system and verifications.
\item After verification the solaris operating system will make a request on
he ZFS volume used as swap partition.
\item The request on the ZFS volume will be translated to an hypercall to the
host operating system for the block on the virtual device.
\item The host operating system will verify the call and translate the request
to a location on the logical volume associated with the virtual machine.
\item The logical volume manager of the host operating system will make a
request to the raid virtual device.
\item The raid layer will arrange the I/O and access the disk.
\item The result will go all the way back.
\end{enumerate}

The disk has an average access time of 18.7 ms but the I/O operation may be
delayed.
We except the result to be at least 100 ms to swap an entire page in due to
the need to schedule the operation in a context where the memory is full and
heavy paging is occuring.

\paragraph{Results}
\begin{table}[h]
\begin{center}
\begin{tabular}{| l | l | l | l | l |}
\hline
Operation 	& Hardware cost & Software cost & Prediction 	& Measured \\ \hline
Page fault	& 18.7 ms	& 81.3 ms 	& 100 ms	& 238 ms\\ \hline
Page fault	& 18.7 ms	& 81.3 ms 	& 100 ms	& 498 ms\\ \hline
Page fault	& 18.7 ms	& 81.3 ms 	& 100 ms	& 8 ms\\ \hline

\end{tabular}
\end{center}
\caption{Page fault service time\label{tab:page-fault}}
\end{table}
% One entire page 786377221 cycles
% One entire page 1643464556 cycles
% One entire page 25242244 fake

We choosed to report our three different results to show that the cost of paging
in a page is really dependent of the workload.
As seen in table \ref{tab:page-fault}, the page fault time is really high and the
variation are considerable.
This is due to other pages which are currently being paged in and out
delaying our page in operation.

The time to access an integer in main memory is about 12ns so 3ns per byte.
The access time of a byte located in a page which needs to be paged is
between 2$\mu$s and 122$\mu$s which is 6000 to 40000 times slower.

\paragraph{Success of Methodology}
Our methodology seems to work and to give interresting results.
The case we simulated is kind of extreme and may not exist on real workloads.
We really overloaded the system, any system subject to this kind of workload
would be unusable.
