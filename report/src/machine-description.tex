\section{Machine Description}
The machine is a virtual machine running on top of a xen hypervisor, the
description mainly concern the hardware of the hypervisor.

\subsection{Processor}
The hypervisor is running an Intel quad core processor of the Sandy Bridge generation.
This is an x86-64 architecture.
The exact model of the CPU is \emph{Intel(R) Core(TM) i5-2500 CPU @ 3.30GHz}. \cite{intel-i5-2500}
The processor integrates the memory controller.

Detailled characteristics of the hardware processor :
\begin{description}
\item[CPU clock frequency] 3300.152 Mhz
\item[L1 cache] 64 KB
\begin{description}
\item[Data cache] 32 KB
\item[Instruction cache] 32 KB
\end{description}
\item[L2 cache] 256 KB
\item[L3 cache] 6 MB
\item[Address sizes] 36 bits physical, 48 bits virtual
\end{description}
The virtual machine is restricted to two virtual processor.

\subsection{Motherboard buses}
The motherboard is an \emph{Intel DQ67OW} \cite{dq670w-motherboard}. It is using an \emph{Intel Q67
Express Chipset}. \cite{q67-chipset}
The processor is connected to the southbridge chipset using a \emph{DMI 2.0} bus which
allow a transfer rate up to 20Gb/s.

\subsection{Ram}
The host system is equipped with 16GB of DDR3.
The memory is running at 1333 Mhz with a CAS of 9.
The memory controller is directly integrated in the CPU, according to the
specification of the processor, the memory bus frequence is 1333Mhz which allow a bandwidth up
to 21 GB/s. \cite{intel-i5-2500}

\subsection{Disk}
\subsubsection{Base hardware}
The system is running on top of 6 hard disk drives.
Four of them are \emph{Hitachi Deskstar P7K500}\footnote{The full reference is
Hitachi HDP725050GLA360}.
One of them is an \emph{Hitachi Deskstar 7K1000.C}\footnote{The full reference
is Hitachi HDS721050CLA362}.
The fifth one is only used for backup purpose so we are not interrested in his
specification.

The P7K500 has the following characteristics \cite{p7k500} :
\begin{description}
\item[Rotational speed] 7200 RPM
\item[Cache size] 16 MB of cache
\item[Size] 500GB
\item[Max Media transfer rate] 1138 Mbit/s
\end{description}
The 7K1000.C has the same characteristics \cite{7K1000.C} excepted for the maximum media transfer
rate which is 1546 Mbit/s due to differences in the architecture of the disks.

\subsubsection{Hypervisor usage of the disks}
These disks are aggregated at the hypervisor level using the linux mdadm
software utility.
The hypervisor root file system runs on top of a RAID 1 array of four disks and
one hot spare.
The data are stored on a software RAID 10 array of 4 disks and one hot spare.
The hypervisor also runs a logical volume manager on top of the RAID10 array to
ease the management.

\subsubsection{Virtual machine disk}
The virtual machine disk is a logical volume.

\paragraph{Read performance}
Any read operation incurs two read a block on two different disks so the
performances are triggered down by the worst disk model.
Without any overload and in a perfect situation, the performance of sequential read
operation would be the sum of the bandwith of the four disks.

\paragraph{Write performance}
For a write operation, any write operation implies to write on the four disks.
Again performance are triggered down by the worst disk model.
Without any overload and in a perfect situation, the performance of sequential
write would be two times the bandwith of the worst disk.

\subsubsection{Performance penality}
%FIXME

\subsection{Network}
The network card is integrated to the moterboard. It's an \emph{Intel® 82579LM Gigabit
Ethernet Controller} which is a gigabit ethernet controller.
The server is connected to a switch which port are running at 100 mb/s.

\subsection{Operating System}
The operating system is Solaris 11.
Output of the uname command :
\begin{verbatim}
SunOS solaris1 5.11 11.0 i86pc i386 i86pc
\end{verbatim}
Content of the /etc/release file :
\begin{verbatim}
                           Oracle Solaris 11 11/11 X86
  Copyright (c) 1983, 2011, Oracle and/or its affiliates.  All rights reserved.
                            Assembled 18 October 2011
\end{verbatim}

\subsection{Hypervisor}
The hypervisor is Xen.
The virtual machine is running in HVM mode and make use of the Intel VT-d and
Intel EPT instructions sets.
Basic instructions doesn't suffer from any overhead as their instructions are
executed natively.
TLB miss are most costly as the processor need to walk through two page tables
to translate the Guest Virtual address to Host Virtual Address and then the Host
Virtual Address to Host Physical Address.
%FIXME
