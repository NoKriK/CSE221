\appendix
\appendixpage
\addappheadtotoc

\section{Remote host\label{sec:app-remotehost}}
The remote server is also a virtual machine.
This virtual machine is running on top of a VMware ESXi hypervisor.
It is running on top off an infrastructure we don't control so we don't know all
the details.

\subsection{Processor}
The hypervisor is running an Intel quad core processor of the Nehalem generation.
This is an x86-64 architecture.
The exact model of the CPU is \emph{Intel(R) Xeon(R) CPU W3565 @ 3.20GHz}.
\cite{intel-xeon-3565}
The processor integrates the memory controller.

Detailled characteristics of the hardware processor :
\begin{description}
\item[Clock frequency] 3200.000 Mhz
\item[L1 cache] 64 KB
\begin{description}
\item[Data cache] 32 KB
\item[Instruction cache] 32 KB
\end{description}
\item[L2 cache] 256 KB
\item[L3 cache] 8 MB
\item[Address sizes] 40 bits physical, 48 bits virtual
\end{description}
The virtual machine is restricted to two virtual processor.

\subsection{Motherboard buses}
Unknown

\subsection{Ram}
The virtual machine has 512MB of virtual memory.
We don't know the hardware configuration.

\subsection{Disk}
The virtual machine is running on top of a 20GiB virtual disk.
The virtual disk is stored on a shared iSCSI LUN.
We won't go further into the detail as we are not benchmarking this part.

\subsection{Network}
The physical server is plugged into a 1GB switch port.
The network is bridged to the physical switch.

\subsection{Operating System}
Linux Debian Squeeze 6.0.4
Output of the uname command :
\begin{verbatim}
Linux sas1 2.6.32-5-amd64 #1 SMP Mon Jan 16 16:22:28 UTC 2012 x86_64 GNU/Linux
\end{verbatim}

\section{Summary table}


\begin{table}[h]
\begin{center}
\begin{tabular}{| l | r | r | r | r |} \hline
Operation 			& Hardware cost 	& Software cost 	& Prediction	& Measured \\ \hline
Reading the clock 	& 25 cycles		& 0 cycles			& 25 cycles 	& 58.555 cycles \\ \hline
Loop 				& 15 cycles 		& 0 cycles 		& 15 cycles 	& 12.487 cycles \\ \hline \hline

Procedure with 0 argument & 2 cycles 	& 8 cycles  		& 10 cycles 	& 14.471 cycles \\ \hline
Procedure with 1 argument  & 2 cycles 	& 9 cycles  		& 11 cycles 	& 14.471 cycles \\ \hline
Procedure with 2 argument  & 2 cycles 	& 10 cycles  		& 12 cycles 	& 14.455 cycles \\ \hline
Procedure with 3 argument  & 2 cycles 	& 11 cycles 		& 13 cycles 	& 14.462 cycles \\ \hline
Procedure with 4 argument  & 2 cycles 	& 12 cycles 		& 14 cycles 	& 14.461 cycles \\ \hline
Procedure with 5 argument  & 2 cycles 	& 13 cycles 		& 15 cycles 	& 14.461 cycles \\ \hline
Procedure with 6 argument  & 2 cycles 	& 14 cycles 		& 16 cycles 	& 14.456 cycles \\ \hline
Procedure with 7 argument  & 2 cycles 	& 15 cycles 		& 17 cycles 	& 16.518 cycles \\ \hline\hline

System call (1st run) & 200 cycles & 600 cycles & 800 cycles & 5000 cycles\\
\hline
System call (average) & 200 cycles & 600 cycles & 800 cycles & 750 cycles\\
\hline\hline

fork() process first time		& 20 cycles & 10000 cycles 	& 10000 cycles& 2257852 cycles \\ \hline
fork() process average		& 20 cycles & 10000 cycles 	& 10000 cycles& 1198519.65 cycles \\ \hline
thr\_create() thread first time	& 20 cycles & 8000 cycles		& 8000 cycles &  2273593 cycles \\ \hline
thr\_create() thread 	average	& 20 cycles & 8000 cycles		& 8000 cycles &  157433.27 cycles \\ \hline\hline


Process context switch 	& 5000 cycles	& 30000 cycles	& 150000 cycles	& 100540.28 cycles \\ \hline
Thread context switch 	& 2000 cycles	& 20000 cycles	& 75000 cycles	& 94764.77 cycles \\ \hline
Pipe overhead		& 200 cycles	& 800 cylces	& 1000 cycles	& 2263.07 cycles \\ \hline

\end{tabular}
\end{center}

\caption{Summary table: CPU operations}

\end{table}

\begin{table}[h]
\begin{center}
\begin{tabular}{| l | r | r | r | r |} \hline
Operation 			& Hardware cost 	& Software cost 	& Prediction	& Measured \\ \hline

512B array	&	4 cycles	&	0 cycle		&	4 cycles	&3.895 cycles	\\ \hline
1KB array	&	4 cycles	&	0 cycle		&	4 cycles	&3.886 cycles		\\ \hline
2KB array	&	4 cycles	&	0 cycle		&	4 cycles	&3.885 cycles		\\ \hline
4KB array	&	4 cycles	&	0 cycle		&	4 cycles	&3.886 cycles		\\ \hline
8KB array	&	4 cycles	&	0 cycle		&	4 cycles	&3.886 cycles		\\ \hline
16KB	array	&	4 cycles	&	0 cycle		&	4 cycles	&3.888 cycles		\\ \hline
32KB array	&	4 cycles	&	0 cycle		&	4 cycles	&3.891 cycles		\\ \hline
64KB 	array	&	12 cycles	&	0 cycle		&	12 cycles	&5.949 cycles		\\ \hline
128KB array	&	12 cycles	&	0 cycle		&	12 cycles	&5.945 cycles		\\ \hline
256KB array	&	12 cycles	&	0 cycle		&	12 cycles	&6.209 cycles		\\ \hline
512KB array	&	28 cycles	&	0 cycle		&	28 cycles	&10.314 cycles		\\ \hline
1MB array	&	28 cycles	&	0 cycle		&	28 cycles	&10.399 cycles		\\ \hline
2MB array	&	28 cycles	&	0 cycle		&	28 cycles	&10.757 cycles		\\ \hline
4MB array	&	28 cycles	&	0 cycle		&	28 cycles	&15.585 cycles		\\ \hline
8MB array	&	45 cycles	&	0 cycle		&	45 cycles	&30.265 cycles		\\ \hline
16MB array	&	45 cycles	&	0 cycle		&	45 cycles	&37.458 cycles		\\ \hline
32MB array	&	45 cycles	&	0 cycle		&	45 cycles	&37.686 cycles		\\ \hline
64MB array	&	45 cycles	&	0 cycle		&	45 cycles	&37.092 cycles		\\ \hline
128MB array&	45 cycles	&	0 cycle		&	45 cycles	&37.619 cycles		\\ \hline \hline








RAM Read 128MB 	& 10000 MB/s	& 0 MB/s	& 10000 MB/s	& 9711.66 MB/s \\
\hline
RAM Write 128MB & 8000 MB/s	& 0 MB/s	& 8000 MB/s	& 7575.58 MB/s \\
\hline \hline

Page fault	& 18.7 ms	& 81.3 ms 	& 100 ms	& 238 ms\\ \hline
Page fault	& 18.7 ms	& 81.3 ms 	& 100 ms	& 498 ms\\ \hline
Page fault	& 18.7 ms	& 81.3 ms 	& 100 ms	& 8 ms\\ \hline



\end{tabular}
\end{center}
\caption{Summary table: RAM operations}

\end{table}






\begin{table}[h]
\begin{center}
\begin{tabular}{| l | r | r | r | r |} \hline
Operation 			& Hardware cost 	& Software cost 	& Prediction	& Measured \\ \hline

%connect time
Local ICMP rtt	& 0 ms	& 0.06 ms & 0.06 ms  & 0.063 ms	\\ \hline
Local TCP rtt	& 0 ms	& 0.012 ms & 0.012 ms	& 0.065 ms\\ \hline
Remote ICMP rtt	& 0.12 ms	& 0.32 ms & 0.56ms  & 1.253 ms\\ \hline
Remote TCP rtt	& 0.12 ms	& 0.32 ms & 0.56ms  & 1.26 ms	\\ \hline\hline


%%peak bandwidth
Local peak bandwidth	&  2161 MB/s	& 23 ms/GB	&  1455 MB/s &  1132MB/s \\
\hline
Remote peak bandwidth 	&  11.25 MB/s	& 0 &  11.25MB/s	&  10.56MB/s \\
\hline\hline


%connection overhead
Local setup 	&  0.12 ms	& 2.48 ms	&  2.5 ms & 0.074 ms \\ \hline
Remote setup	&  0.12 ms	& 2.48 ms	&  2.5 ms&  9.899 ms \\ \hline
Local tear-down 	&  0	& 0.003 ms	&  0.003 ms&  0.017 ms \\ \hline
Remote tear-down	&  0	& 0.003 ms	&  0.003 ms&  0.013 ms \\ \hline





\end{tabular}
\end{center}

\caption{Summary table: Network}

\end{table}












\begin{table}[h]
\begin{center}
\begin{tabular}{| l | r | r | r | r |} \hline
Operation 			& Hardware cost 	& Software cost 	& Prediction	& Measured \\ \hline





% File read time%%%%%%%%%%%%%%
16MB Local Random		& 0.1 ms/block		& 300 ns/block     & 0.1 ms/block	        & 0.022  ms/block \\ \hline
32MB Local Random		& 1.3 ms/block		& 300 ns/block     & 1.3 ms/block	        & 2.157  ms/block \\ \hline
64MB Local Random		& 1.3 ms/block		& 300 ns/block     & 1.3 ms/block	        & 1.357  ms/block \\ \hline
128MB Local Random 		& 8 ms/block		& 300 ns/block     & 8 ms/block	       & 8.980  ms/block \\ \hline
256MB Local Random		& 9 ms/block		& 300 ns/block     & 9 ms/block	       & 9.962  ms/block \\ \hline
512MB Local Random		& 10 ms/block		& 300 ns/block     & 10 ms/block	        & 10.361  ms/block \\ \hline
1024MB Local Random		& 11 ms/block		& 300 ns/block     & 11 ms/block	        & 10.488  ms/block \\ \hline
2048MB Local Random		& 12 ms/block		& 300 ns/block     & 12 ms/block	        & 10.788  ms/block \\ \hline\hline
                                                                                   
16MB Local Sequential		& 0.1 ms/block		& 300 ns/block     & 0.1 ms/block	        & 0.070   ms/block \\ \hline
32MB Local Sequential		& 0.1 ms/block		& 300 ns/block     & 0.1 ms/block	        & 0.717   ms/block \\ \hline
64MB Local Sequential		& 0.65 ms/block		& 300 ns/block     & 0.65 ms/block	        & 0.439   ms/block \\ \hline
128MB Local Sequential 	& 1.3 ms/block		& 300 ns/block     & 1.3 ms/block	        & 6.876   ms/block \\ \hline
256MB Local Sequential	& 1.3 ms/block		& 300 ns/block     & 1.3 ms/block	        & 7.717   ms/block \\ \hline
512MB Local Sequential	& 1.3 ms/block		& 300 ns/block     & 1.3 ms/block	        & 8.687   ms/block \\ \hline
1024MB Local Sequential	& 1.3 ms/block		& 300 ns/block     & 1.3 ms/block	        & 8.721   ms/block \\ \hline
2048MB Local Sequential	& 1.3 ms/block		& 300 ns/block     & 1.3 ms/block	        & 8.826   ms/block \\ \hline\hline

16MB Remote Random	& 0.1 ms/block	& 600 ns/block  & 0.1 ms/block	& 0.026   ms/block \\ \hline
32MB Remote Random	& 0.1 ms/block	& 600 ns/block  & 0.1 ms/block	& 0.025   ms/block \\ \hline
64MB Remote Random	& 0.1 ms/block	& 600 ns/block  & 0.1 ms/block	& 0.026   ms/block \\ \hline
128MB Remote Random 	& 0.1 ms/block	& 600 ns/block  & 0.1 ms/block	& 0.028   ms/block \\ \hline
256MB Remote Random	& 0.1 ms/block	& 600 ns/block  & 0.1 ms/block	& 0.027   ms/block \\ \hline
512MB Remote Random	& 13 ms/block	& 600 ns/block  & 13 ms/block	& 15.401   ms/block \\ \hline
1024MB Remote Random	& 18 ms/block	& 600 ns/block  & 18 ms/block	& 58.903   ms/block \\ \hline\hline

16MB Remote Sequential	& 0.1 ms/block	& 600 ns/block  & 0.1 ms/block        & 0.029    ms/block \\ \hline
32MB Remote Sequential	& 0.1 ms/block	& 600 ns/block  & 0.1 ms/block        & 0.020    ms/block \\ \hline
64MB Remote Sequential	& 0.1 ms/block	& 600 ns/block  & 0.1 ms/block        & 0.019    ms/block \\ \hline
128MB Remote Sequential 	& 0.1 ms/block & 600 ns/block       & 0.1 ms/block        & 0.020    ms/block \\ \hline
256MB Remote Sequential	& 0.1 ms/block & 600 ns/block      & 0.1 ms/block       & 0.019    ms/block \\ \hline
512MB Remote Sequential	& 6 ms/block   & 600 ns/block      & 6 ms/block       & 4.919    ms/block \\ \hline
1024MB Remote Sequential	& 10 ms/block  & 600 ns/block      & 10 ms/block       & 8.725    ms/block \\ \hline\hline 



\end{tabular}
\end{center}

\caption{Summary table: File system (1)}

\end{table}



\begin{table}[h]
\begin{center}
\begin{tabular}{| l | r | r | r | r |} \hline
Operation 			& Hardware cost 	& Software cost 	& Prediction	& Measured \\ \hline





%file cache%%%%%%%%%%%%%%%%%%%%%%%%%%%%%%%%%%%%%%%%%%%%%%%%%%%%%
128MB  & 4GB/s   & 79ms/GB & 3.59GB/s & 1205.06 MB/s \\ \hline
256MB  & 4GB/s   & 79ms/GB & 3.59GB/s & 1182.96 MB/s \\ \hline
384MB  & 4GB/s   & 79ms/GB & 3.59GB/s & 1196.82 MB/s \\ \hline
512MB  & 4GB/s   & 79ms/GB & 3.59GB/s & 1225.79 MB/s \\ \hline
640MB  & 4GB/s   & 79ms/GB & 3.59GB/s & 1253.05 MB/s \\ \hline
768MB  & 4GB/s   & 79ms/GB & 3.59GB/s & 1244.87 MB/s \\ \hline
896MB  & 4GB/s   & 79ms/GB & 3.59GB/s & 1260.24 MB/s \\ \hline
1024MB & 4GB/s   & 79ms/GB & 3.59GB/s & 1256.54 MB/s \\ \hline
1152MB & 4GB/s   & 79ms/GB & 3.59GB/s & 1249.65 MB/s \\ \hline
1280MB & 4GB/s   & 79ms/GB & 3.59GB/s & 1234.26 MB/s \\ \hline
1408MB & 100MB/s & ?       & ?        & 84.5288 MB/s \\ \hline
1536MB & 100MB/s & ?       & ?        & 66.2602 MB/s \\ \hline
1664MB & 100MB/s & ?       & ?        & 86.7509 MB/s \\ \hline
1792MB & 100MB/s & ?       & ?        & 77.2734 MB/s \\ \hline
1920MB & 100MB/s & ?       & ?        & 81.6763 MB/s \\ \hline
2048MB & 100MB/s & ?       & ?        & 83.2122 MB/s \\ \hline
2176MB & 100MB/s & ?       & ?        & 80.2009 MB/s \\ \hline
2304MB & 100MB/s & ?       & ?        & 72.7354 MB/s \\ \hline
2432MB & 100MB/s & ?       & ?        & 78.0652 MB/s \\ \hline\hline

% Contention%%%%%%%%%%%%%%%%%%%%%%%%%%
1 process	& 8 ms/block	& 300 ns/block       & 8 ms/block        & 8.942    ms/block \\ \hline
2 processes	& 16 ms/block	& 300 ns/block       & 16 ms/block       & 2.664    ms/block \\ \hline
4 processes	& 32 ms/block	& 300 ns/block       & 32 ms/block       & 23.693    ms/block \\ \hline
8 processes	& 64 ms/block	& 300 ns/block       & 64 ms/block       & 64.129    ms/block \\ \hline
16 processes	& 128 ms/block	& 300 ns/block       & 128 ms/block      & 121.170    ms/block \\ \hline
32 processes	& 256 ms/block	& 300 ns/block       & 256 ms/block      & 221.781    ms/block \\ \hline



\end{tabular}
\end{center}

\caption{Summary table: File system(2)}

\end{table}




