\appendix
\appendixpage
\addappheadtotoc

\section{Remote host\label{sec:app-remotehost}}
The remote server is also a virtual machine.
This virtual machine is running on top of a VMware ESXi hypervisor.
It is running on top off an infrastructure we don't control so we don't know all
the details.

\subsection{Processor}
The hypervisor is running an Intel quad core processor of the Nehalem generation.
This is an x86-64 architecture.
The exact model of the CPU is \emph{Intel(R) Xeon(R) CPU W3565 @ 3.20GHz}.
\cite{intel-xeon-3565}
The processor integrates the memory controller.

Detailled characteristics of the hardware processor :
\begin{description}
\item[Clock frequency] 3200.000 Mhz
\item[L1 cache] 64 KB
\begin{description}
\item[Data cache] 32 KB
\item[Instruction cache] 32 KB
\end{description}
\item[L2 cache] 256 KB
\item[L3 cache] 8 MB
\item[Address sizes] 40 bits physical, 48 bits virtual
\end{description}
The virtual machine is restricted to two virtual processor.

\subsection{Motherboard buses}
Unknown

\subsection{Ram}
The virtual machine has 512MB of virtual memory.
We don't know the hardware configuration.

\subsection{Disk}
The virtual machine is running on top of a 20GiB virtual disk.
The virtual disk is stored on a shared iSCSI LUN.
We won't go further into the detail as we are not benchmarking this part.

\subsection{Network}
The physical server is plugged into a 1GB switch port.
The network is bridged to the physical switch.

\subsection{Operating System}
Linux Debian Squeeze 6.0.4
Output of the uname command :
\begin{verbatim}
Linux sas1 2.6.32-5-amd64 #1 SMP Mon Jan 16 16:22:28 UTC 2012 x86_64 GNU/Linux
\end{verbatim}
