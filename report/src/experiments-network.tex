\subsection{Network}

\subsubsection{Round trip time}

//NOTE: following is the guide line
Evaluate for the TCP protocol. For each quantity, compare both remote and loopback interfaces. Comparing the remote and loopback results, what can you deduce about baseline network performance and the overhead of OS software? For both round trip time and bandwidth, how close to ideal hardware performance do you achieve? In describing your methodology for the remote case, either provide a machine description for the second machine (as above), or use two identical machines.
//NOTE: above is the guide line

\paragraph{Methodology}
We developped a basic tcp server which accept connections and sends back the
data it receives.
The tcp client connect to the server and does one write and one read of a message of
the size of the Maximum Segment Size (MSS). The measuement time does not include the time it takes for open close socket operation, hand-shake and termination. The RTT we measured is exactly the time it takes for data to go through a round trip from the system to the remote host.
The operation is repeated 10,000 times and the result reported is the average.

For the ICMP measure, we averaged the resulted on 1000 runs.
\begin{verbatim}
ping -n -s 127.0.0.1 56 1000
\end{verbatim}

The second machine is described in the Appendix \ref{sec:app-remotehost}.

\paragraph{Predictions}
The cost of remote request will be high as the packet will travel through :
\begin{enumerate}
\item The Solaris OS
\item The Xen hypervisor through an hypercall.
The Xen hypervisor also does Network Address Translation (NAT)
\item Two or more switches
\item The ESXi hypervisor
\item The linux guest OS
\end{enumerate}

We except an higher cost for TCP as the protocol is statefull and the packet
will be checked against the state tables in the solaris and linux virtual
machine but also on the Xen hypervisor (router/NAT). 

\paragraph{Results}
\begin{table}[h]
\begin{center}
\begin{tabular}{| l | l | l | l | l |}
\hline
Operation 			& Hardware cost 	& Software cost 		& Prediction 		& Measured \\
\hline
Local ICMP rtt		&				&					&				& 		\\
\hline
Remote ICMP rtt		&				&					&				& 		\\
\hline
Local TCP rtt		&				&					&				& 		\\
\hline
Remote TCP rtt		&				&					&				& 		\\
\hline


\end{tabular}
\end{center}
\caption{Round trip time\label{tab:rtt}}
\end{table}

As seen in table \ref{tab:rtt}
\paragraph{Success of Methodology}

 Compare with the time to perform a ping (ICMP requests are handled at kernel level).






\subsubsection{Peak bandwidth}
\paragraph{Methodology}
We measured the time elapsed when transfering 2GB of data.
We developped a tcp server which just drops the packet he receives.
The client send as much data as he can by packet of the size of the MSS.

We ran the test 100 time and averaged the result.

\paragraph{Predictions}
The slowest network switch is switching at theoritical speed of 100Mb/s.
We except the switch to be the bottleneck and to be able to handle only 90\% of
the theoritical 100Mb/s.
Thus we except

\paragraph{Results}
\begin{table}[h]
\begin{center}
\begin{tabular}{| l | l | l | l | l |}
\hline
Operation & Hardware cost & Software cost & Prediction & Measured \\
\hline
Local 	&  cycles		& 10			&  cycles	&  cycles \\
\hline
Remote 	&  cycles		& 10			&  cycles	&  cycles \\
\hline

\end{tabular}
\end{center}
\caption{Peak bandwidth\label{tab:peak-bandwidth}}
\end{table}

As seen in table \ref{tab:Peak bandwidth}
\paragraph{Success of Methodology}
We may have written more data on the socket but it doesn't sound like the
bottleneck ?




\subsubsection{Connection overhead}
\paragraph{Methodology}
\paragraph{Predictions}
\paragraph{Results}
\begin{table} [h]
\begin{center}
\begin{tabular}{| l | l | l | l | l |}
\hline
Operation & Hardware cost & Software cost & Prediction & Measured \\
\hline

Local setup 	&  cycles		& 10			&  cycles	&  cycles \\
\hline
Remote setup	&  cycles		& 10			&  cycles	&  cycles \\
\hline
Local tear-down 	&  cycles		& 10			&  cycles	&  cycles \\
\hline
Remote tear-down	&  cycles		& 10			&  cycles	&  cycles \\
\hline


\end{tabular}
\end{center}
\caption{Connection overhead\label{tab:connection-overhead}}
\end{table}

As seen in table \ref{tab:connection-overhead}

\paragraph{Success of Methodology}
